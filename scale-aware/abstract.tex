\begin{abstract} 
Image segmentation is a key component in many computer vision systems, and it is recovering a prominent spot
in the literature as methods improve and overcome their limitations.
The outputs of most recent algorithms are in the form of a hierarchical segmentation,
which provides segmentation at different scales in a single tree-like structure.
%, usually represented by the so-called ultrametric contour map.
Commonly, these hierarchical methods start from some low-level features, and are not aware of the scale
information of the different regions in them.
As such, one might need to work on many different levels of the hierarchy to find the objects in the scene.
This work tries to modify the existing hierarchical algorithm by improving their alignment, that is, by trying
to modify the depth of the regions in the tree to better couple depth and scale.
%Thus the segmentation trees produced by these methods may result in a suboptimal performance. In this paper, we aim to prove the significance of rescaling the hierarchy, at a global level.
To do so, we first train a regressor to predict the scale of regions using mid-level features.
We then define the anchor slice as the set of regions that better balance between over-segmentation and
under-segmentation.
The output of our method is an improved hierarchy, re-aligned by the anchor slice.
%To speed up the optimization, we perform entropy-driven sampling over the segmentation tree.
To demonstrate the power of our method, we perform comprehensive experiments on the BSDS500 dataset, which shows that our method, as a post-processing step, can significantly improve the quality of the hierarchical segmentation representations.
We also prove that the improvement generalizes well across different algorithms, with a low computational cost.
\yh{Add Pascal coco???}
\end{abstract}
