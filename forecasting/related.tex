\section{Related work}
\textbf{Example-based texture synthesis.} Techniques of example-based
texture synthesis can be broadly categorized into four categories:
feature-oriented image reconstruction~\cite{Heeger:95,
  Portilla:2000:IJCV, random:phase}, Markov Random Fields (MRFs)
methods~\cite{noncausal:tip98, Zalesny05}, tile-based
methods~\cite{Cohen:2003:wang, Liu:2004:NTA}, and
neighborhood-based methods~\cite{Efros:sig2001, Kwatra:tog:2005,
  Kwatra:2003}. The first group learn statistics of a brand of
carefully designed features and coerce the synthetic images to have
the same, e.g. color histograms~\cite{Heeger:95}, wavelet
features~\cite{Portilla:2000:IJCV}, and frequency
spectrum~\cite{random:phase}.  The main difficulty lies in designing a
common set of features that is able to capture the essence of all
kinds of textures.  The second group believe that textures are
instances of MRFs. Parameters of some forms of MRFs are estimated from
the texture examples and new textures are then sampled from the
model. Multi-scale neighborhood-based MRFs are learned
in~\cite{noncausal:tip98} and pairwise clique-based MRFs
in~\cite{Zalesny05}. The third group generate textures by copying
pixels or patches from the exemplar inputs~\cite{Efros:1999,
  Efros:sig2001, Kwatra:2003, Kwatra:tog:2005}. While unlike the first
two groups to provide a key for texture analysis, this group are often
more efficient and tend to work for a larger variety of textures. The
last group assemble new textures out of a set of (rectangular) tiles
croped from example images. It is efficient, but alignment of tiles to
texture elements is often difficult~\cite{Cohen:2003:wang,
  Liu:2004:NTA, dai:facade:iccv13}.

\textbf{Texture recognition.}  Our work is related also related to
texture and material recognition. Efforts have been made mainly on
designing effective features to distinguish textures from different
material categories, including statistics of filter
response~\cite{texton:2001}, the joint intensity distributions within
a compact neighborhood ~\cite{material:pami:09, sorted:texture}, and
the combination of multiple texture features~\cite{material:ijcv13}.
Our system is solving a different task: learning the synthesizability
of texture images instead of their classes. Thus, we need to design
new features for this new task. 
