
% italics
\newcommand{\etal}[0]{~\emph{et al.}~}
\newcommand{\ie}[0]{~\emph{i.e.}~}
\newcommand{\eg}[0]{~\emph{e.g.}~}
\newcommand{\cf}[0]{~\emph{c.f.}~}

% citations
\newcommand{\etalcite}[1]{\emph{et al.}~\shortcite{#1}}
\newcommand{\twocite}[1]{\shortcite{#1}}
\newcommand{\onecite}[1]{\shortcite{#1}}

% sizes
\def\figwidth{\textwidth}
\def\figwidthSmaller{0.90\textwidth}
\def\figwidthsingle{0.7\textwidth}
\def\figwidthdouble{0.47\textwidth}
\def\figwidthtripple{0.30\textwidth}
\def\figwidthquad{0.22\textwidth}


% maths
\newcommand{\argmax}{\operatornamewithlimits{argmax}} 
\newcommand{\argmin}{\operatornamewithlimits{argmin}} 

% Mukta's Defines
\definecolor{muktamxsizecolor}{RGB}{150,150,255}
\newcommand{\matx}[1]{{\tt #1}}
\newcommand{\inv}[1]{#1^{-1}}
\newcommand{\vect}[1]{{\mathbf #1}}
\newcommand{\mati}[1]{{\tt #1}^{-1}}
\newcommand{\braces}[1]{\left\{ #1 \right\}}
\newcommand{\paren}[1]{\left( #1 \right)}
\newcommand{\bracket}[1]{\left[ #1 \right]}
\newcommand{\muktamatrixsize}[3]{\ensuremath \underset{\textcolor{muktamxsizecolor}{#2 \times #3}}{#1} }
\newcommand{\muktasmallmatrixsize}[3]{\ensuremath \underset{\textcolor{muktamxsizecolor}{{\tiny #2 \times #3}}}{#1} }
\newcommand{\muktamatrix}[1]{\ensuremath \begin{matrix} #1 \end{matrix}}
\newcommand{\muktabmatrix}[1]{\ensuremath \begin{bmatrix} #1 \end{bmatrix}}
\newcommand{\muktaBmatrix}[1]{\ensuremath \begin{Bmatrix} #1 \end{Bmatrix}}
\newcommand{\muktavmatrix}[1]{\ensuremath \begin{vmatrix} #1 \end{vmatrix}}
\newcommand{\muktaVmatrix}[1]{\ensuremath \begin{Vmatrix} #1 \end{Vmatrix}}
\newcommand{\muktapmatrix}[1]{\ensuremath \begin{pmatrix} #1 \end{pmatrix}}
\newcommand{\muktasmallmatrix}[1]{\ensuremath \begin{smallmatrix} #1 \end{smallmatrix}}

% comments
\newcommand{\mycomment}[1]{{\textcolor{blue}{#1}}}


\newcommand{\syn}[1]{$#1^\prime$}
\newcommand{\synmn}[3]{$#1^\prime_{#2^\prime, #3^\prime}$}


\def\d{\mathcal{D}}
\def\dtrain{\mathcal{D}^\text{train}}
\def\ntrain{{N}^\text{train}}
\def\ntest{{N}^\text{test}}
\def\dtest{\mathcal{D}^\text{test}}
\def\feat{\vect{f}}
\def\simi{\text{sim}}
\def\numsamples{\inf}
%\newcommand{\argmax}{\operatornamewithlimits{argmax}} 
%\newcommand{\argmin}{\operatornamewithlimits{argmin}} 
\def\enprox{{EnProx}\xspace}
\def\wt{{\sc WT}\xspace}


\newcommand{\efet}[1]{{\vect{#1}}}
\newcommand{\cfet}[1]{{\vect{#1}}}
\newcommand{\ctfet}[1]{\bar{\vect{#1}}}